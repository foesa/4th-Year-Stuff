% !TEX TS-program = pdflatex
% !TEX encoding = UTF-8 Unicode

% This is a simple template for a LaTeX document using the "article" class.
% See "book", "report", "letter" for other types of document.

\documentclass[11pt]{article} % use larger type; default would be 10pt

\usepackage[utf8]{inputenc} % set input encoding (not needed with XeLaTeX)

%%% Examples of Article customizations
% These packages are optional, depending whether you want the features they provide.
% See the LaTeX Companion or other references for full information.

%%% PAGE DIMENSIONS
\usepackage{geometry} % to change the page dimensions
\geometry{a4paper} % or letterpaper (US) or a5paper or....
\geometry{margin=1in} % for example, change the margins to 2 inches all round
% \geometry{landscape} % set up the page for landscape
%   read geometry.pdf for detailed page layout information

\usepackage{graphicx} % support the \includegraphics command and options

% \usepackage[parfill]{parskip} % Activate to begin paragraphs with an empty line rather than an indent

%%% PACKAGES
\usepackage{booktabs} % for much better looking tables
\usepackage{array} % for better arrays (eg matrices) in maths
\usepackage{paralist} % very flexible & customisable lists (eg. enumerate/itemize, etc.)
\usepackage{verbatim} % adds environment for commenting out blocks of text & for better verbatim
\usepackage{subfig} % make it possible to include more than one captioned figure/table in a single float
% These packages are all incorporated in the memoir class to one degree or another...

%%% HEADERS & FOOTERS
\usepackage{fancyhdr} % This should be set AFTER setting up the page geometry
\pagestyle{fancy} % options: empty , plain , fancy
\renewcommand{\headrulewidth}{0pt} % customise the layout...
\lhead{}\chead{}\rhead{}
\lfoot{}\cfoot{\thepage}\rfoot{}

%%% SECTION TITLE APPEARANCE
\usepackage{sectsty}
\allsectionsfont{\sffamily\mdseries\upshape} % (See the fntguide.pdf for font help)
% (This matches ConTeXt defaults)

%%% ToC (table of contents) APPEARANCE
\usepackage[nottoc,notlof,notlot]{tocbibind} % Put the bibliography in the ToC
\usepackage[titles,subfigure]{tocloft} % Alter the style of the Table of Contents
\renewcommand{\cftsecfont}{\rmfamily\mdseries\upshape}
\renewcommand{\cftsecpagefont}{\rmfamily\mdseries\upshape} % No bold!

%%% END Article customizations

%%% The "real" document content comes below...

\title{CS7091 - Industrial Internship Midpoint Report}
\author{173324649 - Efeosa Louis Eguavoen}
%\date{} % Activate to display a given date or no date (if empty),
         % otherwise the current date is printed 

\begin{document}
\maketitle

\section{SMART Goals}
For the course of my Internship, I decided to set myself a mixture of both technical goals and soft goals. I decided this as I thought my growth as an engineer is not only bound by the depth of my technical skills but also by my soft skills as I need to develop my interpersonal skills also.

\subsection{Technological Goals}
\subsubsection{Goal 1: Learn 1 new technology in asynchronous programming and successfully implement it in a task.}
\textbf{Specific}: This goal is specific to my development as an engineer as I feel this is an area I'm unfamiliar with as I haven't used it at all in my college work.  This is something that I wish to work on during the course of my internship so I can improve my level of skill in this area and thus strengthen my engineering skill as a result. 
\\ \textbf{Measurable}: This goal can be measured by the completion of a pull request to the main code base in which I have utilised a technology in asynchronous programming in the code I have written. 
\\ \textbf{Achievable}:This goal is achievable as from talking with my team lead,  our team uses asynchronous process in the code base we're responsible for,  so completing this goal should be within the scope of my internship.  There are numerous asynchronous technologies that are deployed in Hubspot so for me to complete this goal, I should be able to achieve this with the help of my team.
\\ \textbf{Relevant}: The relevance of this skill is that companies often have to deal with tasks asynchronously due to the fact they may take quite some time to complete,  so we can't let our code just hang while it completes.  Improving my skill in this field is definitely of benefit to me as an engineer as it makes me more well rounded. 
\\ \textbf{Time Bound}: This goal can be completed as soon as I'm given a task that requires me to use asynchronous processes.  I have set myself a deadline of the end of April to complete this goal. 
\subsubsection{Goal 2: Develop my coding style to be compatible with the world of work to reduce time how long my pull requests are in review by understanding the coding practices at Hubspot}
\textbf{Specific}: This goal is for me to improve my coding style and the way I code to be more professional in nature and thus more compatible with the world of work. This goal is specific as I've specified why I wish to achieve this and how this can be achieved. 
\\ \textbf{Measurable}: This goal is measurable by comparing the length of time it takes for my pull requests to be merged to the main code base.  This goal can also be measured by the number of comments pertaining to coding style in pull requests changing.
\\ \textbf{Achievable}:This goal can be achieved as there's a document at Hubspot of coding practices and architectural patterns to be deployed when coding.  By my reading,  understanding and most importantly,  implementation of this document I can achieve this goal.  My team members reinforcement on maintaining these standards in day to day work and pull requests can help me achieve this also.
\\ \textbf{Relevant}: This is a relevant goal to have as the readability and maintainability of a code base is very important especially when a codebase grows in size and becomes more and more difficult to pay off the technological debt.  We can help avoid this somewhat by maintaining standards and practices in how we should code,  to reduce the amount of time necessary to maintaining legacy code and increase the ease of understanding of people that may reuse your code later.
\\ \textbf{Time Bound}: This goal is something that I'll continuously be working on over the course of my internship.  I've set myself a deadline of the end of my internship to be able to fully incorporate the coding style.
\subsection{Soft Goals}
\subsubsection{Goal 1: Be able to receive feedback and also give back constructive feedback in order to increase my growth as an engineer by giving me instruction in areas I may be lagging behind, whilst being able to give my team lead direction in how best I learn and thus grow}
\textbf{Specific}: This goal is related to how I receive feedback as an engineer and how I take the feedback,  whilst also being able to support my growth as an engineer by giving me team lead instruction in how best I learn as an engineer so that can be accommodated to increase my growth in turn. 
\\ \textbf{Measurable}:This goal is measurable through the feedback I receive from my team lead.  Measuring soft skills is often more difficult as they aren't as easily quantified,  but for this specific goal it can be measured via the feedback I receive from my team lead in terms of how I've acted given previous feedback and how I've implemented previous feedback.
\\ \textbf{Achievable}: This is an achievable goal as I have weekly 1:1's with my team lead to discuss things and see how I'm getting along.  Through these 1:1's I can get and give feedback frequently enough that I can achieve this goal. 
\\ \textbf{Relevant}: This skill has plenty relevance as an important aspect of growth is via taking feedback and implementing this feedback to plug the holes so to speak.  Equally,  being able to give quality feedback can be very important as you can help nurture those around you and show those around you the best way to interact with you to fast track your growth as we all learn differently and thus need different styles of interaction.
\\ \textbf{Time Bound}: This goal is something I will continuously be working on thorough the course of my internship as learning to take feedback and giving good constructive feedback requires a long time to learn how to do well. 
\subsubsection{Goal 2: Increase my networking skills by having at least 10 1:1's with people on and outside my team to widen my network and learn more about other people's progression in Hubspot}
\textbf{Specific}:The goal here is to increase the amount of people I know in the industry by talking and interacting with numerous people in Hubspot and learning about the path they took to reach the different roles and such they've gained. This goal is specific to me ability to network and narrowed down to an essential part of networking,  talking to new people. 
\\ \textbf{Measurable}: This goal is measurable in the amount of 1:1's and new connections I succeed in making over my time in Hubspot.  If I can successfully have at least 10 1:1's with different people in Hubspot,  I think I'll have achieved my goal of increasing my networking skills and also widening my network.
\\ \textbf{Achievable}: This goal is achievable as in Hubspot there are multiple slack channels dedicated to arranging 1:1's coffee chats with anyone that's interested in partaking.  Also,  my team lead can assist me in arranging meetings with engineers and team leads on other teams also.
\\ \textbf{Relevant}: This goal has relevance as whilst your technical skill as an engineer is important,  it's difficult to show how good an engineer you are when you meet someone for the first time.  Instead the most important skill here is how you network and how well you are able to sell yourself.  Also, by talking to people that have been in the industry for a while,  you can learn from them and their mistakes to streamline your progression in your career. 
\\ \textbf{Time Bound}: I intend to complete at least one 1:1  a week between now and the end of my internship which should total to more than 10.  I've set myself a hard deadline of the 2nd week of May. 
\section{Reflective Diary}
\subsection{Week 1}
For the first week, we were focused mainly on getting setup. We went through numerous onboarding workshops the first week to teach us about the HubSpot product, the different aspects of the HubSpot product and a general introduction to our different departments. During the first week, I also met with my team lead, Gary MacElhinney and we talked a little about our team and what products we own and maintain. I created a slack channel to talk with my fellow interns and such in the first week also as there wasn't many engineering interns in total so I thought it'd be a good idea to keep in touch. 
\\\\ 
Reflection: At the time the onboarding felt somewhat un-necessary, but looking back at it now, it gave me a lot of context into what HubSpot actually is and why the work I do normally is important. Starting in a pandemic definetly affected how I felt about the whole onboarding as I felt somewhat disconnected from the rest of the interns but making the intern channel helped subside this feelings. 
\subsection{Week 2}
This week was more focused on the engineering side. We were given an engineering project that would introduce us to all the different tools and processes HubSpot uses in their engineering teams. The project had both a mixture of front end and back end development so we got to interface with the whole stack. I also got to meet the other members of my team and get a more indepth explanation of the repositories and such that we own. As part of the onboarding project, the final task is to send out a newbie intro - essentially an email about yourself that includes a link to do a 1:1 with anyone that's interested. Due to this, I got a chance to have a meeting with another engineer in the Chicago branch of Hubspot. 
\\\\
Reflection: The onboarding engineering project was very dense and a little confusing at the time. I felt I could've completed it quicker if I had asked for help quicker and not spent so much time trying to figure stuff out myself as I had a team to rely on. But I now reference the onboarding project all the time as now those tools I was introduced to have more context of their usefullness and how to implement them. Doing the 1:1 with someone not on my team was great and helped me to grow my confidence and move towards accomplishing my goal of growing my network.

\subsection{Week 3}
This week I was assigned my first actual task. My task was to implement a utility class to create a standardised way to output multicurrency codes as we need to make multicurrencies more standardised to complete a large project of making items and such imported from other companies editable within HubSpot. To complete this task, I learnt about dependancy injection and Guice, a technology that allows us to do dependancy injection. I did a team tech talk on blockchain and decentralised finance in front of my entire team this week also. On Friday after work, our team played some video games together so we could get to know each other better.
\\\\
Reflection: The first task was rather daunting as learning about a new complex system whilst trying to code in a way to reduce future maintenance whilst maintaing readability was difficult. I felt like I didn't communicate as best I could to get myself unblocked and moving forward as I didn't want to ask silly questions. If In could change anything I would've asked more questions and taken more time to learn about the tools available to me to help me get unblocked. Doing the team tech talk was very important for me to increase my confidence in public speaking but also teaching my team about new technologies felt like I was having some real input. The team bonding on the Friday really went a long way to increasing my familiarity with my team mates and helped me understand the culture more at HubSpot.
\subsection{Week 4}
This week I created my first pull request for my inital task. There was a lot of comments and feedback about my code from my team that I spent most of the week working through so the code would be in a good state that it could be merged, especially There's a big emphasis on test driven development in my team so there was numerous iterations and testing involved before the code was in a good state. I also finally met with the manager of my team, Ricky. 
\\\\
Reflection: The whole PR process felt kind of personal initally as my code was under a lot scrutinzed. But I realized now that the whole point of the intense scrutiny isn't to knock you down but to encourage better coding practices such as consistent naming of parameters, utilizing certain design patterns and thorough testing to prevent unseen bugs as when the code goes to production, it can affect customers negatively which should be avoided as much as possible. 

\subsection{Week 5}
This week I moved onto my next task. As part of our teams future mission, we need to standardise fields across the products we own. As a part of this, we need to standardise when a new multi-currency gets added to a person's account and how this is represented. To do this I needed to add learn a new technology in asynchronous programming, Kafka and create a new Kafka consumer. (Go more into depth about this later). This week I also had my monthly review with my team lead Gary. We discussed about some of my strong points such as asking questions and being willing to take on new tasks and such. He also gave me some areas to improve on such as being more forthcoming with updates on my issues and becoming more independant in my work by being able to find the answers to my questions myself.
\\\\
Reflection: The review gave me some really actionable feedback. The positive feedback helped me reinforce those traits, whilst the constructive criticism helped me to see some of my flaws. I realised that since I'm more used to work by myself, I tend to go off for hours and code and then just present when I'm done, which is against our teams motto of building together. Keeping the other members of your team in the loop of what you're doing is advantageous as they can suggest approaches you may not have thought of and can also see mistakes you're making as you're making them, which reduces on time in PR. As a result of this, work can be done quicker resulting in a higher turn over of work in a period of time.

\subsection{Week 6}
At the start of this week, I was asked to pivot from working on the multicurrencies to working on Hublets based work as out team was currently in Code Red(Maintenance mode essentially) and our team was trying to exit it as soon as possible to work on new features instead. Currently all data goes through the North America data centres but as part of GDPR, customers have the right to chose if their data can leave the US. Also there's high latency for requests as they have to travel so far to get a response. This is where Hublets come, they help solve GDPR concerns whilst also reducing latency. The task I was given was to redo all the webhooks for Shopify to target a new load balancer that would then redirect the data to the right Hublet, be it the one in North America or Europe. To complete this job I had to write a backfill job which is comprised of 2 main parts; fixing all the old data and then changing how any new data is created. I learnt about Jackson, a json processor for java that can create classes and such for from templates that you create. This helps to standardise the code and reduce the amount of future work necessary as classes are auto-generated from your templates. 
\\\\
Reflection: Taking on the feedback from the previous week definetly helped me to increase the rate at which I worked on this task this week. Initally it was hard pivoting from one task to another but my team helped me to gain a footing in the Hublets work as they had been working on it for some time. Leaning on your team and using them as a resource is incredibly useful and helped me to be able to make progress much quicker than if I had done what I was doing before and just coding by myself and not giving updates and such. They unblocked me when I provided regular updates to my issues and we did some pair programming work when necessary. An efficent team can certainly increase the productivity of yourself as an engineer as multiple people working in collaboration can always solve problems much quicker than if you were working on something by yourself.
\section{Technology Management Processes}

More text.

\end{document}
