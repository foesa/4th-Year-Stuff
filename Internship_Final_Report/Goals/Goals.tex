\chapter{Goals}

\section{Technological Goals}
\subsubsection*{Goal: Learn 1 new technology in asynchronous programming and successfully implement it in a task.}
\textbf{Goal Achievement: }  In terms of the accomplishment of this goal,  I feel I completed it and might have even somewhat overachieved.  I built a consumer that uses the Kafka asynchronous programming technology and also used HubSpot's in house queueing system TQ2 in a subsequent task to bring it to completion.\\
\textbf{Goal Evaluation: } This goal was somewhat difficult to accomplish as using Kafka isn't something that is necessary in every task.  To drive this goal to completion,  I specifically asked my team lead if I could work on a task that would allow me to gain some exposure to async technologies.  One of the initial problems was how complicated Kafka was as a system and how in-depth the documentation was.  I experienced some paralysis with getting started initially as there was so much to consider and I was over-thinking the task at hand.  To overcome this,  I sought some guidance from other members of my team that had previously worked on Kafka and asked them for some guidance in how to start.  This enabled me to be able to get a better idea of what the task I was doing actually was and how to use Kafka to accomplish the task. \\  
\textbf{Success Factors:}  The biggest thing that helped me to get this goal to completion was the help of my team mates. By making it clear with my team lead that I had an interest in async technology and wanted to work with it,  my team lead actively assigned me tasks that allowed me to learn more about it.  Using my team mates expertise and experience as a resource helped me drive this goal to completion as they gave me the blueprint to using Kafka and what pitfalls to watch out for which made the process much smoother. \\
\textbf{Goal Re-evaluation:} If I could change this goal,  I would've tried to get more exposure to TQ2,  which was the in house software used.  I used it only a few times during my time at Hubspot but I feel like I didn't get as deep an understanding into how it works like I did for Kafka. 


\subsubsection*{Goal 2: Develop my coding style to be compatible with the world of work to reduce time how long my pull requests are in review by understanding the coding practices at Hubspot}
\textbf{Goal Achievement: } In terms of accomplishment,  I don't feel I accomplished this goal to the fullest extent but I definitely made a lot of progress on this goal.  I learnt how to write clean,  maintainable code and about how to write code with others in mind. \\
\textbf{Goal Difficulty Evaluation:}  Overall,  I think this goal might have been slightly difficult as changing your coding style drastically, quickly is rather difficult and takes a long time to do than I had at the internship.  One of the biggest barriers to achievement in this goal was I had to rethink about how I write code and instead use patterns and such that were widely used as the code would need to be maintained by someone else who I may never meet months down the line,  so the code would need to be able to explain itself.  To solve this,  I actively took notes on my pull requests to see what I was getting pulled up on multiple times and made the active choice to prevent that when I was completing tasks.  By the end of my intership,  I realised the average time a task spent in pr had reduced by roughly 25\% and the comments in relation to coding style were much less frequent.  Whilst I couldn't incorporate everything into my coding style,  I could see I made serious progress and achieved my goal overall.\\
\textbf{Success Factors:} The biggest driver of success for me was the comments on my pull requests and reading 'Clean Code' by Robert Cecil. Using the core concepts in the book gave me the foundation to build off as well as alerting me to some of the mistakes I had been making up to this point.  The pull-request comments acted as a sort of positive reinforcement to give me a gentle remind me about certain things I had been slacking on. \\
\textbf{Goal Re-evaluation:} One thing I would change here is I would expand this out to testing also.  I learnt a lot about testing while at HubSpot but I feel I didn't learn as much about how to write good tests and such as much as I would've liked and this was sort of my weak point. 


\section{Soft Skill Goals}
\subsubsection*{Goal 1 : Be able to receive feedback and also give back constructive feedback in order to increase my growth as an engineer by giving me instruction in areas I may be lagging behind, whilst being able to give my team lead direction in how best I learn}
\textbf{Goal Achievement: } I think I achieved this goal completely.  By the end of my internship,  I had created a positive feedback loop between myself and my team lead and other members on my team,  thus creating an space that was open and judgement free that enabled us all to grow as engineers. \\
\textbf{Goal Difficulty Evaluation:} I think this goal was absolutely necessary and even perpetuated into my other goals as being able to give and receive feedback is important as an engineer in terms of growth as you need to be able to change your style and learn new things constantly due to the nature of the industry.  The goal wasn't difficult to achieve as at HubSpot there's a culture of growth and positivity around giving each other feedback and especially on my team as my team lead felt responsible for the growth of each member of the team, making it easier for me to achieve this goal.  One challenge I faced was accepting that the comments on pull-requests aren't there to attack your ability to code but are instead there to help you to write better code.  By being able to receive this type of feedback better and re-orientating how I thought about it,  it enabled me to realise faults in my own work and work on fixing them.  In terms of giving feedback,  I had weekly personal meetings with my team lead to discuss things in general.  In these meetings I focused on discussing how I work as a person and how my team lead can work alongside this to give me feedback and help me grow.  These meetings accelerated the achievement of this goal as I could continually practice giving good feedback as I could continually tune what I should say as I could see weekly what was and wasn't working.  \\  
\textbf{Success Factors:} The biggest drivers for success on this task were the weekly meetings with my team lead.  Having this pre-allocated time every week to give each other feedback made it that bit easier to discuss things as I knew that this space was made specifically to talk which created a mental break for myself \\
\textbf{Goal Re-evaluation:} If I could go back and change anything about this goal,  I would expand it so that I could also be able to give feedback on other engineers pr's as I felt this was something I didn't do enough and could improve on.  This would also have the added side-effect of being able to learn from the senior engineers in terms of how they write code,  which I could incorporate into my own style. 


\subsubsection*{Goal 2: Increase my networking skills by having at least 10 1:1’s with people on and outside my team to widen my network and learn more about other peoples progression in Hubspot}
\textbf{Goal Achievement: } I overachieved in terms of this goal.  I managed to meet over 15 people in the time I was there and grew my Linkedin connections by over 50 which I felt made this goal an overwhelming success.  \\
\textbf{Goal Difficulty Evaluation:} I felt like this was an easier goal to accomplish in relation to my other goals as setting up meetings with people at HubSpot wasn't difficult as there was already numerous things in place at HubSpot for meeting new people and as an intern,  the intern team did coffee meets with interns based on different teams around the world.  I didn't have  any obstacles in achieving this goal.  \\  
\textbf{Success Factors:} The coffee chats with the other interns were important in driving this goal to fruition as the platform to meet new people and such was already in place,  I just had to use it. \\
\textbf{Goal Re-evaluation:} If I could make an improvement on this goal if I could go into the past,  I would ask to talk with people that maybe aren't in the engineering teams so I could get a better idea of HubSpot as a company overall. 

\section{Major Takeaways}
From setting goals to achieve during my internship,  they helped me to guide my focus and also encourage new behaviours.  By actively trying to change how I code,  I found myself thinking more about how other people would perceive my code and what pitfalls they would point out.  Goal setting also helped me manage my tasks and prioritize them in-terms of what would drive my goals forward and what would take away from them.  I also found a sort of motivation from setting goals as I had something bigger I was working towards and felt a sense of pride when I took major steps towards fulfilling these goals.